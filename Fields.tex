\section{Fields}

\begin{definition}
    A \textit{field} $(F,+,\cdot)$ is a non-empty set $F$ along with two binary operations $'+'$(called addition) and $'\cdot'$(called multiplication) on $F\times F$ satisfying the following conditions for all $a,b,c \in F$
    \begin{enumerate}[(i)]
        \item $F$ is closed under $+$ and $\cdot$
        \item Associative laws: $(a+b)+c=a+(b+c)$ and $a\cdot(b\cdot c)=(a\cdot b)\cdot c$ 
        \item Commutative laws: $a+b=b+a$ and $a\cdot b=b\cdot a$ for all $a,b\in F$.
        \item Distributive law: $a\cdot(b+c)=a\cdot b+a\cdot c$ . 
        \item $\exists$ elements in $F$, $0$ and $1$, $0 \neq 1$ such that $a+0=a$ and $a\cdot 1=a$.
        \item For every $a\in F$, there exists an an additive in $F$, denoted $-a$, such that $a+(-a)=0$.
        \item For every $a\neq 0$ in $F$, there exists a multiplicative element in $F$, denoted $a^{-1}$, such that $aa^{-1}=1$.
    \end{enumerate}
\end{definition}

\vspace{2mm}

\begin{exercise}
    Verify that $\mathbb{R}$, $\mathbb{Q}$ and $\mathbb{C}$ are fields
\end{exercise}
\begin{proof}

Defining $'+'$ and $'\cdot'$ as $normal$ addition and multiplication, it is quite trivial that $\mathbb{R}$, $\mathbb{Q}$ and $\mathbb{C}$ satisfy properties $i-vii$. So they are fields.
\end{proof}

\begin{exercise}
    If $a,b \in F$ then $a\cdot b=0 \implies a=0 $ or $b=0$ 
\end{exercise}
\begin{proof}
Assume that $a\neq =0$ and $b \neq 0$. By property $vii$ $\exists$ $a^{-1}$  s.t. $a^{-1} \cdot$ 
$ a=1$. 

\begin{align*}
    a \cdot b &= 0 \\
    (a^{-1} \cdot a) \cdot b &=0 \\
    1 \cdot b &=0 \\
    b &=0 \\
\end{align*}
This contradicts the assumption that $b=0$.

So $a\cdot b=0 \implies a=0 $ or $b=0$ 
\end{proof}

\begin{definition}
    A \textit{finite field} is a field with a finite set of elements. The number of elements in a finite field is called its \textit{order}.
\end{definition}
\subsection{Modular Operation}
\begin{definition}
    Let $a,b$ and $m > 1$, $a,b,m \in \mathbb{Z}$. Then $$a \equiv b(\textrm{mod } m)$$
    if $m|a-b$ i.e $m$ divides $a-b$
\end{definition}
Given integers $a, m > 1$, by division algorithm we have $a = mq + b$ where $b$
is uniquely determined by $a, m$ and $0 \leq b \leq m-1$. So every $a \in \mathbb{Z}$ is congruent to exactly one of $0, 1, 2, .., m-1$ mod$(m)$.

\begin{exercise}
    If $a \equiv b(\textrm{mod } m)$ and $c \equiv d(\textrm{mod } m)$ then show that
    \begin{itemize}
        \item $a+c \equiv b+d(\textrm{mod } m)$
        \item $a-c \equiv b-d(\textrm{mod } m)$
        \item $a\cdot c \equiv b\cdot d(\textrm{mod } m)$
    \end{itemize}
\end{exercise}
\begin{proof}
    \because \phantom{o} $a \equiv b(\textrm{mod } m) \implies m|a-b$ and $a \equiv b(\textrm{mod } m) \implies m|c-d$, 
    \begin{align*}
        m &|(a-b)+(c-d)\\ 
        m &|(a+c)-(b+d) \\ 
        a+c &\equiv b+d(\textrm{mod } m)
    \end{align*}
    Similarly $a-c \equiv b-d(\textrm{mod } m)$
    
    By division algorithm  $a = mq + b$ and  $c = mp + d$ $$a \cdot c = (mq+b)\cdot(mp+d)$$
    $$a \cdot c = m^{2}pq+m(pb+qd)+bd$$ $$ \cdot c = m(mpq+pb+qd)+bd$$ $$m| 
    ac-bd$$ $$a\cdot c \equiv b\cdot d(\textrm{mod } m)$$
    Hence proved.
\end{proof}

\begin{exercise}
    Let $p$ be a prime.
    \begin{itemize}
        \item Show that $ \binom{p}{j} \equiv 1 (\textrm{mod } m)$ for any $j$ s.t. $1 \leq j \leq p-1$
        \item Show that $ \binom{p-1}{j} \equiv (-1)^{j} (\textrm{mod } m)$ for any $j$ s.t. $1 \leq j \leq p-1$
    \end{itemize}
\end{exercise}

\subsection{Finite Fields}
\begin{definition}
    For $m>1$, we define $Z_{m}={0,1,2..,m-1}$. $+$ and $\cdot$ are defined as 
    $$a+b=(a+b)(\textrm{mod } m)$$ $$a\cdot b=(ab)(\textrm{mod } m)$$ 
\end{definition}

\begin{exercise}
    Prove that $Z_{m}$ is a field iff $m$ is prime.
\end{exercise}
\begin{proof}
    p
\end{proof}

\begin{theorem}
    For $n\in\mathbb{N}$, consider the set $\mathbb{Z}_n$ with addition and multiplication defined modulo $n$, that is, $\overline{a}+\overline{b}=\overline{a+b}$ and $\overline{a}\cdot\overline{b}=\overline{ab}$ for $\overline{a}, \overline{b}\in\mathbb{Z}_n$. $\mathbb{Z}_n$ is a field if and only if $n$ is a prime.
\end{theorem}
\begin{proof}
If $n$ is not a prime, then there exist $a,b\in\mathbb{N}$ both less than $n$ such that $ab=n$, that is, $\overline{a}\cdot\overline{b}=\overline{0}$. As the group $\mathbb{Z}_n$ under addition has identity $\overline{0}$, we see that $\mathbb{Z}_n$ cannot be a field by \ref{multiplyByZero}.

For prime $n$, $\mathbb{Z}_n$ is a field as for any $a\not\in \overline{0}$, $(a,n)=1$ and thus a modular multiplicative inverse exists for every element of $\mathbb{Z}_n\setminus\{0\}$ (Recall \ref{BezoutsLemma}).
\end{proof}

This field, called the \textit{prime field} of order $n$, is denoted $\mathbb{F}_n$.

\vspace{2mm}
Let $F$ be a field. For $a\in F, n\in\mathbb{N}$, we denote $a+a+\cdots+a$ ($n$ times) as $na$ and $aa\cdots a$ ($n$ times) as $a^n$.

\subsection{Characteristic of a Field}

\begin{definition}
    Let $F$ be a field. The smallest positive integer $n$ such that $n1=0$ is called the \textit{characteristic} of $F$ and is denoted $\charac F$. If no such $n$ exists, we say that $F$ has characteristic $0$.
\end{definition}

Note that if $\charac F=n$, then $na=0$ for all $a\in F$ ($na=n(1a)=(n1)a=0$).

\begin{theorem}
    Let $F$ be a finite field. Then $\charac F$ is prime.
\end{theorem}
\begin{proof}
    On the contrary, assume that $n=\charac F$ is composite, that is, $n=ab$ for some $a,b\in\mathbb{N}$, $a,b>1$. We have $n1=0$, that is, $(a1)(b1)=0$. Then \ref{multiplyByZero} implies that $a1=0$ or $b1=0$. As $a,b<n$ and $n$ is the smallest positive integer such that $n1=0$, this is a contradiction. Thus, $n$ must be prime.
\end{proof}

\begin{theorem}
    Let $F$ be a finite field. Then the order of $F$ is equal to $p^n$ for some prime $p$ and $n\in\mathbb{N}$.
\end{theorem}
\begin{proof}
    Let $\charac F=p$. Then since $1$ has order $p$ in the group $(F,+)$, $p$ divides the order of $F$.
    
    \vspace{1mm}
    Let $q\neq p$ be another prime dividing the order of $F$. By \ref{CauchyTheorem}, there exists an element of order $q$ in $(F,+)$, that is, there is some non-zero $a$ such that $qa=0$. We also have $pa=0$ because $p = \charac F$. As $p$ and $q$ are distinct primes, $(p,q)=1$.
    
    \vspace{1mm}
    By \ref{BezoutsLemma}, there exist $m,n\in\mathbb{Z}$ such that $mp+nq=1$. We then have $mp(a)+nq(a)=1(a)$ which implies $0=m(pa)+n(qa)=a$. This is a contradiction.
    
    \vspace{1mm}
    Thus, $p$ is the only prime that divides the order of $F$.
\end{proof}

\begin{definition}
    Fields $F$ and $G$ are \textit{isomorphic} if there is a bijection $\varphi:F\to G$ such that $\varphi(x+y)=\varphi(x)+\varphi(y)$ and $\varphi(xy)=\varphi(x)\varphi(y)$ for all $x,y\in F$. Such a map is called an \textit{isomorphism}.
\end{definition}

\begin{theorem}
    Given any prime power $q$, there exists a unique field of order $q$ (up to isomorphism).
\end{theorem}

We omit the proof of the above theorem.

\vspace{2mm}
Given the above, we unambiguously denote the field of order $q$ as $\mathbb{F}_q$.


\clearpage